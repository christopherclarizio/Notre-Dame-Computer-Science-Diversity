\documentclass[letterpaper]{article}

\usepackage{graphicx}
\usepackage{hyperref}
\usepackage[margin=1in]{geometry}

\begin{document}

% Title -----------------------------------------------------------------------

\title{Reading 04: Diversity At Notre Dame}
\date{February 12, 2017}
\author{Christopher Clarizio{\textless}cclarizi@nd.edu{\textgreater}}

\maketitle

% Overview --------------------------------------------------------------------

\section*{Overview}

\paragraph{}

For this reading I created two very similar scripts to extract data 
about diversity within the classes of 2013 to 2018 within the department 
of Computer Science at Notre Dame. One script (gender.sh) extracted data 
from a provided file (demographics.csv)about the number of Men vs. Women 
studying Computer Science in each class from 2013 to 2018. In to do this
the script called a function countgender with two arguments, year and 
gender, and printed out the number of Men in a year and the number of 
Women in a year. The function returned the number of Men and the Number 
of Women by using a pipeline that was constructed out of curl (to get 
the file) cut (to extract the field for the given year), grep (to search 
for the lines that were either Men or Women) and wc (to determine the 
number of lines containing either Men or Women). The second script 
(ethnic.sh) performed largely the same task in almost exactly the same 
way. The difference is that ethnic.sh searched not by gender but instead 
by ethnicity. In performing this task I was unfortunately not suprised 
to learn that overwhelmingly the majority of people studying Comptuer 
Science at Notre Dame are caucasian and that to a slightly lesser degree 
male. Furthermore, while the number of men and women studying Computer 
Science at Notre Dame has grown the proportion of Men to Women has 
remained largely the same. In a similar vein, while the number of people 
studying Computer Science at Notre Dame has increased overall, the 
proportion of caucasian students to minorities has swelled over this 
time period. The takeaway for me is that while Notre Dame's Computer 
Science department has grown it has continued to be majority caucasian 
males.

% Methodology -----------------------------------------------------------------

\section*{Methodology}

\paragraph{}

In order to process the data inside of demographics.csv I wrote two 
shell scripts that operated in largely the same way. The first script 
(gender.sh) used a loop to iterate through the years 2013 to 2018. In 
each iteration it used echo to display the year, and the output of a 
function countgender called twice with the current year and either M 
(for male) or W for (female). This resulted in the number of male and 
female students studying Computer Science to be diplayed. The function 
countgender operates very simply, it sets a variable column equal to 
the column containing the gender information for that year 
((year-2013)+1), as well as the variable gender to either M or F. It 
then uses curl to get the demogrpahics.csv file pipes its output into 
cut to get the information from the proper columns, grep to look for 
lines that match the desired gender (M or F) and finally wc to count the 
number of lines containing that specific gender resulting in the number 
of either men or women from that specific year. The second script 
(ethnic.sh) funcitons almost identically the difference being instead of 
looking for the number of a specified gender it instead looks for a specified 
ethnicity.

% Analysis --------------------------------------------------------------------

\section*{Analysis}

\begin{center}
\begin{tabular}{ |c|c|c|c|c|c|c|c| }
 \hline
 Year & Caucasian & Oriental & Hispanic & African & Native & Multiple &  Undeclared \\
 \hline
 2013 & 43 & 7 & 7 & 3 & 1 & 2 & 0 \\
 \hline
 2014 & 43 & 5 & 4 & 2 & 1 & 1 & 0 \\
 \hline
 2015 & 47 & 9 & 10 & 4 & 1 & 1 & 2 \\
 \hline
 2016 & 53 & 9 & 9 & 1 & 7 & 0 & 0 \\
 \hline
 2017 & 60 & 12 & 3 & 5 & 5 & 6 & 0 \\
 \hline
 2018 & 91 & 8 & 12 & 3 & 4 & 8 & 0 \\
 \hline
\end{tabular}
\end{center}

\begin{center}
\begin{tabular}{ |c|c|c| }
 \hline
 Year & Women & Men \\
 \hline
 2013 & 14 & 49 \\
 \hline
 2014 & 12 & 44 \\
 \hline
 2015 & 16 & 58 \\
 \hline
 2016 & 19 & 60 \\
 \hline
 2017 & 26 & 65 \\
 \hline
 2018 & 36 & 90 \\
 \hline
\end{tabular}
\end{center}

\paragraph{}

\begin{enumerate}

\item{}The overall trend in gender balance at Notre Dame within the 
department of computer science has shifted towards parity. That is the 
gender balance has shifted from three and a half men per women in the 
class of 2013 to only two and a half men per women in the class of 2018.

\item{}Over time ethnic diversity in the Computer Science and 
Engineering program at Notre Dame has gotten worse. That is the 
number of minority students has remained largely the same while the 
number of caucasian students in the class of 2018 has more than 
doubled from the number of caucasian students in the class of 2013.

\end{enumerate}

\paragraph{}

% Discussion ------------------------------------------------------------------

\section*{Discussion}

\paragraph{}

\begin{enumerate}

\item{}In my opinion the department of comptuer science's main goal 
should be to attract as many of the most talented and qualified students 
as possible. In pursuit of this goal I believe that the department is 
most likely missing out on some of the best students if it is limited to 
mainly caucasian males. In pursuit of this goal I do believe that the 
department working to increase diversity would be a good thing. However, 
I believe that the department in seeking to increase diversity should 
keep in mind the diversity of the University as a whole. That is bear in 
mind that the problem of diversity is not limited to the department but 
the university as a whole and this may prevent the department from being 
able to increase its diversity as the University does not have a hugely 
diverse pool of students to draw from. In regards to whether the 
technology industry should work to increase diversity my opinion is 
almost identical to mine about the department increasing diversity. That 
is they should in order to attract the most competent people, however, 
that may be difficult as the pool of qualified individuals may not be 
diverse thanks to university computer science programs not being 
diverse.

\item{}I believe that not only the Computer Science and Engineering 
department but the university as a whole provides a welcoming 
environment for all students. This however, is a difficult question for 
me to answer as I am an individual and cannot hope to speak for all 
students at the university.

\item{}The most significant challenge that I have faced in the Computer 
Science and Engineering program is a lack of connection to my peers. 
That is while the first year engineering program was helpful in 
introducing me to students studying other engineering disciplines it did 
not introduce me to any other students studying computer science and 
engineering which I feel is very important as these are the people that 
I will be studying with mostly for my remaining time at the university.
I believe that the university could do a better job in this regard by 
making modifications to the first year engineering class so that I am 
able to work with my peers earlier or allowing us to take a computer 
science specific class as a freshman.

\end{enumerate}

\end{document}
